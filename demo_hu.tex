\documentclass[10pt,a4paper]{article}

\usepackage[utf8]{inputenc}
\usepackage[a4paper]{geometry}
\usepackage{lmodern}
\usepackage[scaled]{helvet}
\usepackage[T1]{fontenc}
\def\magyarOptions{defaults=hu-min,footnote=huplain,afterindent=force-yes}
\PassOptionsToPackage{hyphenation=composite}{magyar.ldf}
\usepackage[magyar]{babel}
\usepackage{cals}
\usepackage{linebylinepar}

\overfullrule=10pt

\makeatletter

\def\encouragehyphenation{%
  \pretolerance-1  % Always consider hyphenating the paragraph.
  \hyphenpenalty1  % Hyphenating a word is not that bad.
  \exhyphenpenalty2  % Hyphenating a word with a hyphen is not that bad.
}

% Like \struct, but allows automatic hyphenation on both sides.
\def\xstrut{\leavevmode\nobreak\hskip0pt\strut\nobreak\hskip0pt\relax}

% Like \alignL, but doesn't justify.
\newcommand\alignF{%
  \dimen0=\cals@paddingR \cals@paddingR=\dimen0 plus 1fill\relax
  \dimen0=\cals@paddingL \cals@paddingL=\dimen0 \relax}

%** Similar to \raggedright, but adapts to \alignF.
\def\raggedrightF{\let \\\@centercr \@rightskip \rightskip}%

%\def\solA{%
%  \advance\hsize-\cals@paddingL \advance\hsize-\cals@paddingR}

%** The beseline. Justified.
\def\solA#1{\alignL\cell{\xstrut#1\xstrut\par}}%
%** Left-aligned. More line breaks expected than with \solA.
\def\solB#1{\alignF\cell{\xstrut#1\xstrut\par}\alignL}%
%** With 99% probability, produces the same result as \solB.
\def\solC#1{\alignF\cell{\raggedrightF\encouragehyphenation\par
  \xstrut#1\xstrut\par}\alignL}%
%** Produces more line breaks than \solB, but most probably still not
%** enough.
\def\solD#1{\alignF\cell{\sloppy\encouragehyphenation\par
  \xstrut#1\xstrut\par}\alignL}%
%** Produces great results most of the time, but it may fail with an
%** unintelligible error message occasionally.
\def\solE#1{\alignF\cell{\linebylinepar{\xstrut#1\xstrut}}\alignL}%

\makeatother

\begin{document}

\makeatletter
\renewcommand*\familydefault{\sfdefault}
\normalfont

%\def\testtext{megszentségteleníthetetlenségeskedéseitekért, autóbusz`-állomás foo bar foo bar foo bar foo bar foo bar foo bar foo bar}
%\def\testtext{hello world}
\def\testtext{megszentségteleníthetetlenségeskedéseitekért, autóbusz`-állomás megszentségteleníthetetlenségeskedéseitekért}
\def\testkuns{Kunszentmárton, autóbusz`-állomás}

\begin{calstable}
\colwidths{{141pt}{141pt}}
\cals@paddingL=2pt
\cals@paddingR=2pt
\cals@paddingT=2pt
\cals@paddingB=2pt
\thead{
  \bfseries
  \alignC
  \brow
  \cell{\xstrut head1}
  \cell{\xstrut head2}
  \erow
  \mdseries
}
\brow \solA{\testtext} \alignL \cell{\xstrut xoo} \erow
\brow \solB{\testtext} \alignL \cell{\xstrut xoo} \erow
\brow \solC{\testtext} \alignL \cell{\xstrut xoo} \erow
\brow \solD{\testtext} \alignL \cell{\xstrut xoo} \erow
\brow \solE{\testtext} \alignL \cell{\xstrut xoo} \erow
\brow \solC{\testkuns} \alignL \cell{\xstrut xoo} \erow
% This is an example demonstrating that \solE is not maximally greedy:
% it's possible to hyphenate ál-lo-más.
\brow \solE{\testkuns} \alignL \cell{\xstrut xoo} \erow
\end{calstable}

\end{document}
